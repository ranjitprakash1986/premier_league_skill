% Options for packages loaded elsewhere
\PassOptionsToPackage{unicode}{hyperref}
\PassOptionsToPackage{hyphens}{url}
%
\documentclass[
]{article}
\usepackage{amsmath,amssymb}
\usepackage{lmodern}
\usepackage{iftex}
\ifPDFTeX
  \usepackage[T1]{fontenc}
  \usepackage[utf8]{inputenc}
  \usepackage{textcomp} % provide euro and other symbols
\else % if luatex or xetex
  \usepackage{unicode-math}
  \defaultfontfeatures{Scale=MatchLowercase}
  \defaultfontfeatures[\rmfamily]{Ligatures=TeX,Scale=1}
\fi
% Use upquote if available, for straight quotes in verbatim environments
\IfFileExists{upquote.sty}{\usepackage{upquote}}{}
\IfFileExists{microtype.sty}{% use microtype if available
  \usepackage[]{microtype}
  \UseMicrotypeSet[protrusion]{basicmath} % disable protrusion for tt fonts
}{}
\makeatletter
\@ifundefined{KOMAClassName}{% if non-KOMA class
  \IfFileExists{parskip.sty}{%
    \usepackage{parskip}
  }{% else
    \setlength{\parindent}{0pt}
    \setlength{\parskip}{6pt plus 2pt minus 1pt}}
}{% if KOMA class
  \KOMAoptions{parskip=half}}
\makeatother
\usepackage{xcolor}
\usepackage[margin=1in]{geometry}
\usepackage{color}
\usepackage{fancyvrb}
\newcommand{\VerbBar}{|}
\newcommand{\VERB}{\Verb[commandchars=\\\{\}]}
\DefineVerbatimEnvironment{Highlighting}{Verbatim}{commandchars=\\\{\}}
% Add ',fontsize=\small' for more characters per line
\usepackage{framed}
\definecolor{shadecolor}{RGB}{248,248,248}
\newenvironment{Shaded}{\begin{snugshade}}{\end{snugshade}}
\newcommand{\AlertTok}[1]{\textcolor[rgb]{0.94,0.16,0.16}{#1}}
\newcommand{\AnnotationTok}[1]{\textcolor[rgb]{0.56,0.35,0.01}{\textbf{\textit{#1}}}}
\newcommand{\AttributeTok}[1]{\textcolor[rgb]{0.77,0.63,0.00}{#1}}
\newcommand{\BaseNTok}[1]{\textcolor[rgb]{0.00,0.00,0.81}{#1}}
\newcommand{\BuiltInTok}[1]{#1}
\newcommand{\CharTok}[1]{\textcolor[rgb]{0.31,0.60,0.02}{#1}}
\newcommand{\CommentTok}[1]{\textcolor[rgb]{0.56,0.35,0.01}{\textit{#1}}}
\newcommand{\CommentVarTok}[1]{\textcolor[rgb]{0.56,0.35,0.01}{\textbf{\textit{#1}}}}
\newcommand{\ConstantTok}[1]{\textcolor[rgb]{0.00,0.00,0.00}{#1}}
\newcommand{\ControlFlowTok}[1]{\textcolor[rgb]{0.13,0.29,0.53}{\textbf{#1}}}
\newcommand{\DataTypeTok}[1]{\textcolor[rgb]{0.13,0.29,0.53}{#1}}
\newcommand{\DecValTok}[1]{\textcolor[rgb]{0.00,0.00,0.81}{#1}}
\newcommand{\DocumentationTok}[1]{\textcolor[rgb]{0.56,0.35,0.01}{\textbf{\textit{#1}}}}
\newcommand{\ErrorTok}[1]{\textcolor[rgb]{0.64,0.00,0.00}{\textbf{#1}}}
\newcommand{\ExtensionTok}[1]{#1}
\newcommand{\FloatTok}[1]{\textcolor[rgb]{0.00,0.00,0.81}{#1}}
\newcommand{\FunctionTok}[1]{\textcolor[rgb]{0.00,0.00,0.00}{#1}}
\newcommand{\ImportTok}[1]{#1}
\newcommand{\InformationTok}[1]{\textcolor[rgb]{0.56,0.35,0.01}{\textbf{\textit{#1}}}}
\newcommand{\KeywordTok}[1]{\textcolor[rgb]{0.13,0.29,0.53}{\textbf{#1}}}
\newcommand{\NormalTok}[1]{#1}
\newcommand{\OperatorTok}[1]{\textcolor[rgb]{0.81,0.36,0.00}{\textbf{#1}}}
\newcommand{\OtherTok}[1]{\textcolor[rgb]{0.56,0.35,0.01}{#1}}
\newcommand{\PreprocessorTok}[1]{\textcolor[rgb]{0.56,0.35,0.01}{\textit{#1}}}
\newcommand{\RegionMarkerTok}[1]{#1}
\newcommand{\SpecialCharTok}[1]{\textcolor[rgb]{0.00,0.00,0.00}{#1}}
\newcommand{\SpecialStringTok}[1]{\textcolor[rgb]{0.31,0.60,0.02}{#1}}
\newcommand{\StringTok}[1]{\textcolor[rgb]{0.31,0.60,0.02}{#1}}
\newcommand{\VariableTok}[1]{\textcolor[rgb]{0.00,0.00,0.00}{#1}}
\newcommand{\VerbatimStringTok}[1]{\textcolor[rgb]{0.31,0.60,0.02}{#1}}
\newcommand{\WarningTok}[1]{\textcolor[rgb]{0.56,0.35,0.01}{\textbf{\textit{#1}}}}
\usepackage{graphicx}
\makeatletter
\def\maxwidth{\ifdim\Gin@nat@width>\linewidth\linewidth\else\Gin@nat@width\fi}
\def\maxheight{\ifdim\Gin@nat@height>\textheight\textheight\else\Gin@nat@height\fi}
\makeatother
% Scale images if necessary, so that they will not overflow the page
% margins by default, and it is still possible to overwrite the defaults
% using explicit options in \includegraphics[width, height, ...]{}
\setkeys{Gin}{width=\maxwidth,height=\maxheight,keepaspectratio}
% Set default figure placement to htbp
\makeatletter
\def\fps@figure{htbp}
\makeatother
\setlength{\emergencystretch}{3em} % prevent overfull lines
\providecommand{\tightlist}{%
  \setlength{\itemsep}{0pt}\setlength{\parskip}{0pt}}
\setcounter{secnumdepth}{-\maxdimen} % remove section numbering
\ifLuaTeX
  \usepackage{selnolig}  % disable illegal ligatures
\fi
\IfFileExists{bookmark.sty}{\usepackage{bookmark}}{\usepackage{hyperref}}
\IfFileExists{xurl.sty}{\usepackage{xurl}}{} % add URL line breaks if available
\urlstyle{same} % disable monospaced font for URLs
\hypersetup{
  pdftitle={Premier league skill analysis},
  hidelinks,
  pdfcreator={LaTeX via pandoc}}

\title{Premier league skill analysis}
\author{}
\date{\vspace{-2.5em}}

\begin{document}
\maketitle

\hypertarget{objective}{%
\subsection{Objective:}\label{objective}}

Simulating the performance in the 2022-2023 EPL season for the teams and
using a latent skill estimate for determining the ranking of the teams
by the latent skill level. Validating this against the actual EPL
standings at the end of the season.

We begin the analysis by importing the necessary packages for the
analysis.

\begin{Shaded}
\begin{Highlighting}[]
\NormalTok{knitr}\SpecialCharTok{::}\NormalTok{opts\_chunk}\SpecialCharTok{$}\FunctionTok{set}\NormalTok{(}\AttributeTok{echo =} \ConstantTok{TRUE}\NormalTok{)}
\FunctionTok{library}\NormalTok{(tidyverse)}
\end{Highlighting}
\end{Shaded}

\begin{verbatim}
## -- Attaching core tidyverse packages ------------------------ tidyverse 2.0.0 --
## v dplyr     1.1.0     v readr     2.1.2
## v forcats   1.0.0     v stringr   1.5.0
## v ggplot2   3.4.1     v tibble    3.1.7
## v lubridate 1.8.0     v tidyr     1.2.0
## v purrr     0.3.4     
## -- Conflicts ------------------------------------------ tidyverse_conflicts() --
## x dplyr::filter() masks stats::filter()
## x dplyr::lag()    masks stats::lag()
## i Use the ]8;;http://conflicted.r-lib.org/conflicted package]8;; to force all conflicts to become errors
\end{verbatim}

\begin{Shaded}
\begin{Highlighting}[]
\FunctionTok{library}\NormalTok{(rstan)}
\end{Highlighting}
\end{Shaded}

\begin{verbatim}
## Loading required package: StanHeaders
## 
## rstan version 2.26.14 (Stan version 2.26.1)
## 
## For execution on a local, multicore CPU with excess RAM we recommend calling
## options(mc.cores = parallel::detectCores()).
## To avoid recompilation of unchanged Stan programs, we recommend calling
## rstan_options(auto_write = TRUE)
## For within-chain threading using `reduce_sum()` or `map_rect()` Stan functions,
## change `threads_per_chain` option:
## rstan_options(threads_per_chain = 1)
## 
## Do not specify '-march=native' in 'LOCAL_CPPFLAGS' or a Makevars file
## 
## Attaching package: 'rstan'
## 
## The following object is masked from 'package:tidyr':
## 
##     extract
\end{verbatim}

\begin{Shaded}
\begin{Highlighting}[]
\FunctionTok{library}\NormalTok{(rstantools)}
\end{Highlighting}
\end{Shaded}

\begin{verbatim}
## This is rstantools version 2.2.0
\end{verbatim}

\begin{Shaded}
\begin{Highlighting}[]
\FunctionTok{library}\NormalTok{(cowplot)}
\end{Highlighting}
\end{Shaded}

\begin{verbatim}
## 
## Attaching package: 'cowplot'
## 
## The following object is masked from 'package:lubridate':
## 
##     stamp
\end{verbatim}

\begin{Shaded}
\begin{Highlighting}[]
\FunctionTok{library}\NormalTok{(lubridate)}
\FunctionTok{library}\NormalTok{(dplyr)}
\FunctionTok{rstan\_options}\NormalTok{(}\AttributeTok{auto\_write =} \ConstantTok{TRUE}\NormalTok{)  }\CommentTok{\# To save some compiling}
\end{Highlighting}
\end{Shaded}

Loading the data and retaining only the columns that are need for this
analysis.

\begin{Shaded}
\begin{Highlighting}[]
\NormalTok{games\_raw }\OtherTok{\textless{}{-}} \FunctionTok{read\_csv}\NormalTok{(}\StringTok{"../data/epl\_2022\_2023\_matches.csv"}\NormalTok{)}
\end{Highlighting}
\end{Shaded}

\begin{verbatim}
## Rows: 380 Columns: 16
## -- Column specification --------------------------------------------------------
## Delimiter: ","
## chr (9): Day, Date, Time, Home, Score, Away, Venue, Referee, Match Report
## dbl (5): Wk, home_xG, away_xG, home_final_score, away_final_score
## lgl (1): Notes
## 
## i Use `spec()` to retrieve the full column specification for this data.
## i Specify the column types or set `show_col_types = FALSE` to quiet this message.
\end{verbatim}

\begin{Shaded}
\begin{Highlighting}[]
\FunctionTok{head}\NormalTok{(games\_raw)}
\end{Highlighting}
\end{Shaded}

\begin{verbatim}
## # A tibble: 6 x 16
##      Wk Day   Date       Time    Home  home_xG Score away_xG Away  Atten~1 Venue
##   <dbl> <chr> <chr>      <chr>   <chr>   <dbl> <chr>   <dbl> <chr>   <dbl> <chr>
## 1     1 Fri   05-08-2022 20:00 ~ Crys~     1.2 0–2       1   Arse~   25286 Selh~
## 2     1 Sat   06-08-2022 12:30 ~ Fulh~     1.2 2–2       1.2 Live~   22207 Crav~
## 3     1 Sat   06-08-2022 15:00 ~ Tott~     1.5 4–1       0.5 Sout~   61732 Tott~
## 4     1 Sat   06-08-2022 15:00 ~ Newc~     1.7 2–0       0.3 Nott~   52245 St J~
## 5     1 Sat   06-08-2022 15:00 ~ Leed~     0.8 2–1       1.3 Wolv~   36347 Ella~
## 6     1 Sat   06-08-2022 15:00 ~ Bour~     0.6 2–0       0.7 Asto~   11013 Vita~
## # ... with 5 more variables: Referee <chr>, `Match Report` <chr>, Notes <lgl>,
## #   home_final_score <dbl>, away_final_score <dbl>, and abbreviated variable
## #   name 1: Attendance
\end{verbatim}

\begin{Shaded}
\begin{Highlighting}[]
\NormalTok{games }\OtherTok{\textless{}{-}}\NormalTok{ games\_raw }\SpecialCharTok{|\textgreater{}}
  \FunctionTok{mutate}\NormalTok{(}\AttributeTok{date =} \FunctionTok{as.numeric}\NormalTok{(}\FunctionTok{format}\NormalTok{(}\FunctionTok{as.Date}\NormalTok{(games\_raw}\SpecialCharTok{$}\NormalTok{Date, }\AttributeTok{format=}\StringTok{"\%d{-}\%m{-}\%Y"}\NormalTok{), }\StringTok{"\%Y"}\NormalTok{)),}
         \AttributeTok{away\_team =} \FunctionTok{as.factor}\NormalTok{(Away),}
         \AttributeTok{home\_team =} \FunctionTok{as.factor}\NormalTok{(Home),}
         \AttributeTok{away\_xg =} \FunctionTok{as.numeric}\NormalTok{(away\_xG),}
         \AttributeTok{home\_xg =} \FunctionTok{as.numeric}\NormalTok{(home\_xG)}
\NormalTok{         ) }\SpecialCharTok{|\textgreater{}}
  \FunctionTok{select}\NormalTok{(}
\NormalTok{    date, home\_team, home\_final\_score, home\_xg, away\_team, away\_final\_score, away\_xg}
\NormalTok{  )}
\FunctionTok{head}\NormalTok{(games\_raw)}
\end{Highlighting}
\end{Shaded}

\begin{verbatim}
## # A tibble: 6 x 16
##      Wk Day   Date       Time    Home  home_xG Score away_xG Away  Atten~1 Venue
##   <dbl> <chr> <chr>      <chr>   <chr>   <dbl> <chr>   <dbl> <chr>   <dbl> <chr>
## 1     1 Fri   05-08-2022 20:00 ~ Crys~     1.2 0–2       1   Arse~   25286 Selh~
## 2     1 Sat   06-08-2022 12:30 ~ Fulh~     1.2 2–2       1.2 Live~   22207 Crav~
## 3     1 Sat   06-08-2022 15:00 ~ Tott~     1.5 4–1       0.5 Sout~   61732 Tott~
## 4     1 Sat   06-08-2022 15:00 ~ Newc~     1.7 2–0       0.3 Nott~   52245 St J~
## 5     1 Sat   06-08-2022 15:00 ~ Leed~     0.8 2–1       1.3 Wolv~   36347 Ella~
## 6     1 Sat   06-08-2022 15:00 ~ Bour~     0.6 2–0       0.7 Asto~   11013 Vita~
## # ... with 5 more variables: Referee <chr>, `Match Report` <chr>, Notes <lgl>,
## #   home_final_score <dbl>, away_final_score <dbl>, and abbreviated variable
## #   name 1: Attendance
\end{verbatim}

Creating a dictionary of ID for each of the teams in the league. There
are 0 unique teams in this league season.

\begin{Shaded}
\begin{Highlighting}[]
\NormalTok{dictionary\_names }\OtherTok{\textless{}{-}} \FunctionTok{tibble}\NormalTok{(}
  \AttributeTok{team =} \FunctionTok{levels}\NormalTok{(games}\SpecialCharTok{$}\NormalTok{away\_team),}
  \AttributeTok{code =} \DecValTok{1}\SpecialCharTok{:}\DecValTok{20}
\NormalTok{)}

\CommentTok{\# Ensuring the levels are per the alphabetical order}
\FunctionTok{levels}\NormalTok{(games}\SpecialCharTok{$}\NormalTok{away\_team) }\OtherTok{\textless{}{-}} \DecValTok{1}\SpecialCharTok{:}\DecValTok{20}
\FunctionTok{levels}\NormalTok{(games}\SpecialCharTok{$}\NormalTok{home\_team) }\OtherTok{\textless{}{-}} \DecValTok{1}\SpecialCharTok{:}\DecValTok{20}
\end{Highlighting}
\end{Shaded}

Creating new columns that we need for the analysis.

\begin{Shaded}
\begin{Highlighting}[]
\NormalTok{games }\OtherTok{\textless{}{-}}\NormalTok{ games }\SpecialCharTok{\%\textgreater{}\%}
  \FunctionTok{mutate}\NormalTok{(}
    \AttributeTok{winning\_team =} \FunctionTok{case\_when}\NormalTok{(}
\NormalTok{      away\_final\_score }\SpecialCharTok{\textgreater{}}\NormalTok{ home\_final\_score }\SpecialCharTok{\textasciitilde{}}\NormalTok{ away\_team,}
\NormalTok{      away\_final\_score }\SpecialCharTok{\textless{}}\NormalTok{ home\_final\_score }\SpecialCharTok{\textasciitilde{}}\NormalTok{ home\_team}
\NormalTok{    ),}
    \AttributeTok{winning\_score =} \FunctionTok{case\_when}\NormalTok{(}
\NormalTok{      away\_final\_score }\SpecialCharTok{\textgreater{}}\NormalTok{ home\_final\_score }\SpecialCharTok{\textasciitilde{}}\NormalTok{ away\_final\_score,}
\NormalTok{      away\_final\_score }\SpecialCharTok{\textless{}}\NormalTok{ home\_final\_score }\SpecialCharTok{\textasciitilde{}}\NormalTok{ home\_final\_score}
\NormalTok{    ),}
    \AttributeTok{winning\_xg =} \FunctionTok{case\_when}\NormalTok{(}
\NormalTok{      away\_xg }\SpecialCharTok{\textgreater{}}\NormalTok{ home\_xg }\SpecialCharTok{\textasciitilde{}}\NormalTok{ away\_xg,}
\NormalTok{      away\_xg }\SpecialCharTok{\textless{}}\NormalTok{ home\_xg }\SpecialCharTok{\textasciitilde{}}\NormalTok{ home\_xg}
\NormalTok{    ),}
    \AttributeTok{losing\_team =} \FunctionTok{case\_when}\NormalTok{(}
\NormalTok{      away\_final\_score }\SpecialCharTok{\textgreater{}}\NormalTok{ home\_final\_score }\SpecialCharTok{\textasciitilde{}}\NormalTok{ home\_team,}
\NormalTok{      away\_final\_score }\SpecialCharTok{\textless{}}\NormalTok{ home\_final\_score }\SpecialCharTok{\textasciitilde{}}\NormalTok{ away\_team}
\NormalTok{    ),}
    \AttributeTok{losing\_score =} \FunctionTok{case\_when}\NormalTok{(}
\NormalTok{      away\_final\_score }\SpecialCharTok{\textgreater{}}\NormalTok{ home\_final\_score }\SpecialCharTok{\textasciitilde{}}\NormalTok{ home\_final\_score,}
\NormalTok{      away\_final\_score }\SpecialCharTok{\textless{}}\NormalTok{ home\_final\_score }\SpecialCharTok{\textasciitilde{}}\NormalTok{ away\_final\_score}
\NormalTok{    ),}
    \AttributeTok{losing\_xg =} \FunctionTok{case\_when}\NormalTok{(}
\NormalTok{      away\_xg }\SpecialCharTok{\textgreater{}}\NormalTok{ home\_xg }\SpecialCharTok{\textasciitilde{}}\NormalTok{ home\_xg,}
\NormalTok{      away\_xg }\SpecialCharTok{\textless{}}\NormalTok{ home\_xg }\SpecialCharTok{\textasciitilde{}}\NormalTok{ away\_xg}
\NormalTok{    ),}
    \AttributeTok{score\_diff =}\NormalTok{ winning\_score }\SpecialCharTok{{-}}\NormalTok{ losing\_score,}
    \AttributeTok{xg\_diff =}\NormalTok{ winning\_xg }\SpecialCharTok{{-}}\NormalTok{ losing\_xg}
\NormalTok{  ) }\SpecialCharTok{\%\textgreater{}\%}
  \FunctionTok{select}\NormalTok{(}
\NormalTok{    winning\_team, winning\_score, winning\_xg,}
\NormalTok{    losing\_team, losing\_score, losing\_xg, score\_diff, xg\_diff}
\NormalTok{  )}

\CommentTok{\# Removing the rows where there is an NA, draws will be marked with NA}
\NormalTok{games }\OtherTok{\textless{}{-}} \FunctionTok{drop\_na}\NormalTok{(games)}
\end{Highlighting}
\end{Shaded}

We are now left with 284` rows of data.

\hypertarget{building-a-model}{%
\subsection{Building a model}\label{building-a-model}}

The next step in our analysis is to design a Bayesian model that lets us
infer a ranking of teams from the data. This is not an easy task; it
does not fit into any of the usual kinds of problem.

The scores of of the teams are integers, the expected goals (xg) for
each team are floating point numbers. While the scores can only be
positive integers including 0, expected goals can be any positive real
number with 0.0 also a possibility. The higher the score or expected
goals, the better the performance of the team.

Let's now talk about the latent measure of Ranking. It refers to the
relative performance of the team in the league. It is measured in
reality using the point system, where each victory grants the team 3
points, a draw 1 point and none for a loss. However, it is well known
that the points garnered in a season may not be fully in agreement with
the relative skill of the team. The ranking as such are positive
integers with a minimum integer (highest ranking) 1 and the maximum
integer (lowest ranking) equal to the total number of teams
participating in the season. Since the Winning/Losing scores are
positive integers, we can model the game scores as a poisson
distribution with parameter lambda (continuous parameter). However,
taking the ranking of the team is not always a complete indicator of the
game scores. This is because the highest ranked team does not imply they
will have the highest score or highest score difference in every game.
The games can be won with narrowest of score difference or the lowest
possible score. Hence the scores of a team come from a distribution with
parameter lambda that will need to be modeled properly.

Rankings are hard to deal with both computationally (as they live in a
really big discrete space) and in model design (as they do not have any
nice simple parameterization). So often, a good strategy in this sort of
model design setting is to introduce auxiliary variables that let you
easily compute the quantity you care about (in this case, the ranking)
without forcing you to deal with it explicitly.

We could assign a non-negative skill value to each team; then, we could
compute a ranking by simply sorting the teams from highest to lowest
skill value.

Instead of inferring ranking directly, let us infer a (totally
hypothetical) skill value \$ S\_j \textbackslash geq 0\$ for the \(j\)
th team, where \(j = 1, \dots, t\) (\(t\) is the number of teams in the
league, 20 in this study). The \(S_j\) (\(j = 1, \dots, t\)) are now the
unknown variables in our model that we want to infer from data.

We need a way to specify how our observed game result data were
generated using those skill values.

We will treat the `score\_diff` of each game as our observation. Since
`score\_diff` is always a non-negative integer **per game**, we will use
a **Poisson distribution** to model it.

Thus, we have for the \(i\) th game

\[\texttt{score_diff}_i \sim \text{Poisson}(\lambda_i),\]

where \(\lambda_i\) is some **continuous** function of the `skill` of
the `winning\_team` (\$w\_i\$) and the `skill` of the `losing\_team`
(\$l\_i\$). \(\lambda_i\) is allowed to be continuous in a Poisson
random variable.

The mean value of the \texttt{score\_diff} is given by \(\lambda\) since
it is assumed to be Poisson distributed. Thus a higher
\texttt{score\_diff} implies a higher value of
\$\textbackslash lambda\$, which in turn implies a higher difference in
the skill levels of the winning and losing team. We need a functional
representation of \(\lambda\) such that it increases with the increase
in difference in the skill levels of the winning and the losing team.
Consequently the expected score difference also increases. This makes
sense in the the context of the modelling of a football game outcome.

One model that meet this criteria is as follows:

\(\lambda_i = \log\big(1 + \exp(w_i - l_i)\big)\)

\end{document}
